
% P1: driven by increasing computational demands, big data computating 
% frameworks are widely used in many areas. Lots areas face increasing 
% computational demand. But still didn't meet the latency % and fault-tolerance 
% citeria in some areas.
Driven by the increasing computational demands, cluster management systems 
(\eg, \mesos) are already pervasive for deploying many applications. 
Unfortunately, despite much effort, existing systems are still difficult to 
support critical applications (\eg, trading, medical, and military services) 
because these applications naturally demand high-availability and small 
deployment performance overhead. Existing systems typically replicate their 
task controllers in order to tolerate single-point 
failures of this controller and to reschedule applications from their failures. 
However, the applications themselves are still often a single point of failure, 
which inevitably hurt availability.

% P1.1: fault-tolerance. paxos may help, but didn't meet the latency 
% requirements.

% P2: we present a fast, highly-available computing platform by leveraging 
% recent advances in rdma-enabled smr systems. frameworks now automatically 
% enjoy high availability, no needed to implement reliability by themselves. 
% can also address the slow tail problem. feasibility study in two % rdma paxos 
% systems.
This paper proposes \xxx, a cluster management system design that automatically 
provides high-availability to general applications. \xxx's key to achieve fast 
fault-tolerance is a fast \paxos replication protocol that leverages RDMA 
(Remote Direct Memory Access). \xxx runs replicas of the same application with 
a set of controllers, where controllers agree on computational requests 
efficiently with this protocol. Initial evaluation results show that \xxx's 
protocol has low latency overhead compared to the applications' unreplicated 
executions.
% A happy side-effect in \xxx is that it 
% addresses a pervasive latency-tail challenge via its fault-tolerance 
% architecture. 
% and it has the potential to 
% promote the deployment of critical applications in cluster


