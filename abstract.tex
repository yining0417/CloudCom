
% P1: driven by increasing computational demands, big data computating 
% frameworks are widely used in many areas. Lots areas face increasing 
% computational demand. But still didn't meet the latency % and fault-tolerance 
% citeria in some areas.
Driven by the increasing computational demands, cluster management systems 
(\eg, \mesos) are already pervasive for deploying many applications. 
Unfortunately, despite much effort, existing systems are still difficult to 
meet the high requirements of critical applications (\eg, trading, medical, and 
military applications), because these applications naturally require 
high-availability and low performance overhead in deployments. Existing systems 
typically replicate their job controllers so that these controllers are 
highly-available and thus they can handle applications failures. However, 
applications themselves are still often a single point of failure, leaving 
arbitrary unavailabile time windows for themselves.

% P1.1: fault-tolerance. paxos may help, but didn't meet the latency 
% requirements.

% P2: we present a fast, highly-available computing platform by leveraging 
% recent advances in rdma-enabled smr systems. frameworks now automatically 
% enjoy high availability, no needed to implement reliability by themselves. 
% can also address the slow tail problem. feasibility study in two % rdma paxos 
% systems.
This paper proposes the design of \xxx, a cluster management system that 
automatically provides high-availability to general applications. \xxx's key to 
make applications achieve high-availability efficiently is a new \paxos 
replication protocol that leverages RDMA (Remote Direct Memory Access). \xxx 
runs replicas of the same job with a replicas of controllers, and 
controllers agree on job requests efficiently with this protocol.
Evaluation shows that \xxx has low performance overhead in 
both throughput and resonse time compared to an application's 
unreplicated executions.
% A happy side-effect in \xxx is that it 
% addresses a pervasive latency-tail challenge via its fault-tolerance 
% architecture. 
% and it has the potential to 
% promote the deployment of critical applications in cluster


