% \newpage
\section{Evaluation} \label{sec:eval}

% We evaluated an anti-virus scanning server that a b c d e f.

\begin{table}[b]
\footnotesize
\centering
\vspace{-.05in}
\begin{tabular}{lrrr}
{\bf Approach} & {\bf Response time (ms)} \\
\hline\\[-2.3ex]
Native execution                       & 991        \\
\xxx bare framework only                       & 992        \\
\xxx with Helgrind                                   & 1,048     \\
\xxx with drcov                                   & 1,036     \\
\xxx with Helgrind and drcov                       & 1,012        \\
Helgrind only                       & 41,018       \\
drcov only                       & 1,272       \\
\end{tabular}
\vspace{-.05in}
\caption{{\em \clamav's performance running in \xxx.}}
\label{tab:overhead}
\end{table}

We evaluated \xxx on \clamav~\cite{clamav}, a popular anti-virus scanning 
server that scans files in parallel and deletes malicious ones. Our evaluation 
was done on a set of three 
Linux 3.2.14 machines within a 1Gbps bandwidth LAN, and each machine has 2.80 
GHz dual-socket hex-core Intel Xeon with 24 hyper-threading cores and 64GB 
memory. To evaluate performance, we used \clamav's own client utility 
\clamdscan to ask the \clamav server to scan its own source code 
and installation directories, and we measured \clamdscan's 
execution time as the server's response time. The \clamav server spawned eight 
threads to scan these directories in parallel. To avoid network latency, 
\clamdscan was ran within the primary node. Adding network latency will mask \xxx's overhead. Each data point was ran for five times and 
we picked the median value.

To evaluate whether \xxx can transparently run 
analysis tools, we selected two popular tools: one heavyweight tool, the 
Helgrind race detector~\cite{valgrind:pldi}; and one lightweight tool, 
DynamoRio's code coverage tool \drcov~\cite{dynamorio}. 

Table~\ref{tab:overhead} shows the performance results running \clamav in 
\xxx. \xxx's bare replication framework incurred negligible overhead over 
the native execution. When running Helgrind only in one 
replica, \xxx incurred 5.8\% overhead over the native execution, because both 
the other replica node and the primary ran the native executions and they 
reached consensus fast. When running the \drcov tool only or running both 
tools, \xxx incurred only 2.1\% to 3.6\% overhead. This overhead is low because once the \drcov tool 
reached consensus with the primary on the \clamdscan utility's request, the 
primary just processed the request and responsed to \clamdscan regardless of 
\drcov's execution. Considering performance jitters, we viewed \xxx's 
results running with one or both tools equal.

The \drcov tool itself incurred a moderate 28.3\% overhead compared to native 
execution, while Helgrind itself incurred a 40.4X slowdown. 
Table~\ref{tab:overhead}'s results show that \xxx's replication architecture 
masked the huge performance slowdown of Helgrind, then \clamdscan felt that 
\clamav ran efficiently even both the two tools were turned on.

Note that Helgrind is carried in Valgrind, a traditional analysis 
framework that takes the ``fully-coupled" approach. So does \drcov for the 
DynamoRio analysis framework. Overall, this evaluation shows that \xxx is 
efficient, transparent to the 
Helgrind and \drcov tools, and complementary to traditional analysis frameworks 
Valgrind and DynamoRio. In addition, to the best of our knowledge, \xxx's 
evaluation is the first one that has shown to run different types of analysis 
tools together.
