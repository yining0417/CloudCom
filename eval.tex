% \newpage
\section{Evaluation} \label{sec:eval}

% We evaluated an anti-virus scanning server that a b c d e f.

\begin{table}[b]
\footnotesize
\centering
\vspace{-.05in}
\begin{tabular}{lrrr}
{\bf Approach} & {\bf Execution time (ms)} \\
\hline\\[-2.3ex]
Native execution                       & 991        \\
\xxx bare framework only                       & 992        \\
\xxx with Helgrind                                   & 1,048     \\
\xxx with drcov                                   & 1,036     \\
\xxx with Helgrind and drcov                       & 1,012        \\
Helgrind only                       & 41,018       \\
drcov only                       & 1,272       \\
\end{tabular}
\vspace{-.05in}
\caption{{\em \clamav's performance running in \xxx.}}
\label{tab:overhead}
\end{table}

We evaluated \xxx on \clamav~\cite{clamav}, a popular anti-virus scanning 
server that scans files in parallel and deletes malicious ones. Our evaluation 
was done on a set of three 
Linux 3.2.14 machines within a 1Gbps bandwidth LAN, and each machine has 2.80 
GHz dual-socket hex-core Intel Xeon with 24 hyper-threading cores and 64GB 
memory. To evaluate performance, we used \clamav's own client utility 
\v{clamdscan} to ask the \clamav server to scan its own source code 
and installation directories, and we measured \v{clamdscan}'s 
execution time as the server's response time. The \clamav server spawned 8 
threads to process this workload. To avoid network latency, \v{clamdscan} was 
ran within the primary node. Adding network latency would further mask \xxx's 
overhead. To evaluate whether \xxx can transparently run analysis tools, we 
selected two popular tools: one heavyweight tool, the Helgrind race 
detector~\cite{valgrind:pldi}; and one lightweight tool, DynamoRio's code 
coverage tool drcov~\cite{dynamorio}.

Table~\ref{tab:overhead} shows the performance results running \clamav in 
\xxx. \xxx's bare replication framework incurred negligible overhead over 
the native execution. When running Helgrind only in one 
replica, \xxx incurred 5.8\% overhead over the native execution, because both 
the other replica node and the primary ran the native executions and they 
reached consensus fast. When running the drcov tool only or running two tools, 
\xxx incurred only 2.1\% to 3.6\% overhead, because the drcov tool incurred 
moderate overhead (28.3\% compared to native execution), and once the drcov 
tool reached consensus with the primary on the \v{clamdscan} utility's request, 
the primary just processed the request and responsed to \v{clamdscan} 
regardless of drcov's execution.

When running Helgrind only with \clamav, the performance slowdown was 40.4X. 
This result reflects that \xxx's replication architecture masks the huge 
performance slowdown of Helgrind, so that \v{clamdscan} felt that \clamav ran 
efficiently even with both the two analyses were turned on. Note that Helgrind 
is already carried in Valgrind, a traditional analysis framework that takes the 
``fully-coupled" approach. So does drcov for the DynamoRio analysis framework. 
Overall, this evaluation shows that \xxx is efficient, transparent to the 
Helgrind and drcov tools, and complementary to traditional analysis frameworks 
Valgrind and DynamoRio. In addition, to the best of our knowledge, \xxx's 
evaluation is the first one that has shown to run multiple types of analysis 
tools together.
