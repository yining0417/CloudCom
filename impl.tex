 \section{Implementation} \label{sec:impl}

\subsection{Syscall interception in Xen Hypervisor}\label{sec:interception}
A syscall aims at providing userland processes a way to request services 
from the kernel. This includes access storage operations, memory and network 
access, etc. Typical calls are open, close, \read... 

One way to perform a syscall is to use software interrupts. It consists of 
four steps: (1) User process copies the syscall number to execute in the CPU register 
EAX; (2) Execute int 0x80 to generate interrupt 0x80; (3) System calls handling 
routine executed in ring 0; (4) Return the result to the user process.

In order to intercept syscalls within the Xen Hypervisor itself, the first thing to do 
is to disable the Direct Trap optimization that allows the syscalls to bypass the 
hypervisor. Thus, syscalls are forced through the hypervisor and we will be able to 
catch them.  Then, when is issued, the syscall will be intercepted and the information 
we are interested in will be logged. Finally, we will keep on with the regular syscall 
execution.
