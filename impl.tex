% \section{Implementation} \label{sec:impl}

% This section first introduces the implementation details of \xxx's \paxos 
% component, and then presents the implementation choices of the checkpoint 
% component.

% \subsection{\paxos Protocol} \label{sec:paxos}
% % SMR. Paxos. Choose it because: (1) fast. (2) transparent to applications.
% The \paxos consensus component is a critical component to enforce a 
% consistent 
% total order of socket operations from client programs. Although there are 
% already a number of open source \paxos implementations~\cite{concoord, 
% zookeeper, chubby:osdi, libpaxos}, we find it necessary to re-implement a 
% \paxos protocol because it is hard to port their current consensus 
% interface to \xxx's socket consensus interface.
% 
% 
% We have implemented a standard \paxos protocol based on a well-known approach 
% called ``\paxos Made Practical"~\cite{paxos:practical} because it is easy to 
% understand, use, and fast. In normal case, only the primary node is proposing 
% consensus requests, so consensus can be achieved fast. When exceptional cases 
% such as network partitions and node failures occur, a \paxos leader election 
% is 
% invoked to resolve conflicts. Each socket operation from the client is 
% assigned 
% a global, monotonically increasing viewstamp (or \emph{global index}) to 
% identify each checkpoint. Upon consensus on a socket operation, each 
% consensus 
% component persistently stores the operation type, arguments, and global index 
% into Berkeley DB storage on local SSD.



\section{Checkpoint and Rollback Design} \label{sec:checkpoint}

To support synchrouns analysis tools, \xxx provides two checkpoint and rollback 
functions. When an anslysis tool calls the \v{checkpoint()} function, \xxx 
uses its \paxos consensus component to invoke an operation: ``take a checkpoint 
associated with the global index of the latest socket operation". Once a 
consensus is reached, all replicas (including primary) does a checkpoint 
operation as the next consensus operation. This checkpoint operation trade off 
transparency between an analysis tool and \xxx's framework, and may defer 
theprocessing the network requests, but we argue that this operation is worthwhile 
for supporting synchrouns analysis tools in \xxx.

When the tool catches a malicious event and decides to roll back to a 
checkpoint associated with a socket operation index, the analysis can call a 
\v{void rollback(int} \v{index=-1,} \v{bool discard=0)} API provided 
by \xxx. Ignoring the argument in this API means rolling back to the last 
checkpoint within all replicas and discard all inputs since this checkpoint. 
This API works as three steps: (1) current replica uses the \paxos consensus 
component to invoke an operation: ``rollback to a previous checkpoint according 
to \v{index}"; (2) each replica invokes \criu to do the actual rollback 
operation; (3) each replica discards all future inputs after the rollback, if 
\v{discard} is on. This API ensures all replicas to consistently rollback and 
discard malicious inputs, so that following executions can safely bypass the 
malicious events caused by these inputs.

% Second, once an anlysis 
% captures some bad events caued by processing malicious inputs, \xxx must 
% allow % the analysis to invoke an operation to consistently roll back all 
% nodes's % execution states before this malicious input is being processed, 
% and % allow the % analysis to determine re-executing the inputs or discarding 
% the % inputs.

% To transparently checkpoint an application, \xxx's checkpointer component 
% leverages a popular, open source checkpoint-restore tool called 
% \criu~\cite{criu}. This tool has supported CPU registers, memory, etc. However, 
% one key issue still exists: a server constantly serves requests, and 
% checkpointing and restoring TCP stacks on different machines is notoriously 
% difficult. Our trick to avoid this difficulty is that we have observed that 
% server programs always have some idle moments. Thus, a \xxx non-primary replica 
% running an actual execution does a checkpoint periodically (one minute by 
% default) when the replica has no alive socket connections.

% For the checkpoint frequency on analysis nodes, we design it to be high, 
% because: (1) analyses may need to roll back whenever detecting an malicious 
% events; and (2) the time cost by a checkpoint operation does not matter much 
% because the analysis is heavyweight anyway. Thus, a \xxx replica running an 
% analysis performs a checkpoint at the moment the server has finished serving 
% the current batch of socket connections: once the \paxos consensus component on 
% each node finds that currently no socket connections are alive, the replica 
% does a checkpoint. This checkpoint moment is also good for \xxx's rollback API 
% because it helps the analysis get rid of the current socket connections which 
% trigger the malicious events. 


% To meet the second requirement, we create an API that can be called by an 
% analysis tool to invoke a rollback. The API format is: \v{int} 
% \v{rollback(int} % \v{ncheckpoints = 0,} \v{bool skip = false)}. Once the 
% analysis calls this % operation, it invokes a client to send an roll back 
% operation to the primary node with the the list of the last checkpoint's 
% global % indexes. The primary then invokes a consensus on an internal roll 
% back % opration, % with the 
