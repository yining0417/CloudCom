\section{Introduction} \label{sec:intro}

% P1: virtualization is good and desire for fault tolerance in virtualized environment is pervasive.
By simplifying provisioning and allowing multiple servers to be consolidated on a 
small number of physical hosts, virtualization has made low- and middle-end systems 
popular than ever. Nevertheless, the benefits of consolidation come with a hidden cost 
in the form of increased exposure to hardware failure. 

% P2: primary-backup based replication of virtual machines is an attractive technique 
% to ensure fault tolerance in virtualized environments
% two existing approaches:  log-replay is highly architecture-specific and slow for multi-processor

% requiring that the system have a comprehensive understanding of the instruction set being executed and the sources of external events
% reproducing the exact order in which CPU cores access the shared memory

Replication at the Virtual Machine Monitor Layer (VMM) is a promising technique to 
ensure fault tolerance in virtualized environments. It offers seamless failover for the 
complete service state, including everything from the service software to the operating system. 
Currently there are two approaches to primary-backup based replication of virtual machines.
log-replay records input and non-deterministic events of the primary machine 
and have them deterministically replayed on the backup machine to replicate the primary's state 
in case the primary node fails. However, deterministic replay is highly architecture-specific and 
introduces unacceptable performance degradation when applied in multi-core systems.

% P3: the other approach is checkpoint-recovery based replication of virtual machines
% significant and frequent service downtime
% output commit - consistency requirement. But it comes with degraded network performance.
% Unacceptable for online services where clients expect reasonable bandwidth and short response times from the server machines
On the other hand, Checkpoint-recovery based replication of virtual machines is delivered by 
capturing the entire executions state of the VM at relatively high frequency so that changes can 
be reflected to the backup machine nearly instantly. But it suffers from two problems. The first one 
is significant service interruption. The second one is poor network performance.

% P4: Our key insight is that a consensus protocol (e.g. Paxos) to enforce the same input on the service software between 
% the primary machine and the backup nodes is enough to ensure fault-tolerance.
% Here, the input refers to the network interaction from client to the service on the primary machine.

% Note: limitation: cannot handle non-deterministic events.


% P5: However, traditional Paxos is slow
% Fortunately, RDMA opens new opportunity for mitigating consensus latency
However, Paxos consensus is notoriously difficult to be fast. 
To agree on an input, traditional consensus protocols invoke at least one 
message round-trip between two replicas. Fortunately, Remote Direct Access Memory (RDMA) opens new opportunity.

% P6: We present a fast, fault-tolerant system.
% We intercept at the networking level and invoke RDMA-based Paxos protocol.

% P7: benefits: fast (short response time), service is highly available (VM does not need to be suspended) and general (requiring no modification on the service software) 

% P8: We also tackle nondeterminism - output check protocol

% P9: tail-tolerance

% P10: rest of the paper
