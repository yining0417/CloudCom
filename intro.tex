\section{Introduction} \label{sec:intro}

% P1: lots of applications require various big data frameworks. Trading, fraud 
% detection, aviation, medical, military. But existing systems can not meet the 
% high latency and availability requriements of these applications.
Driven by the drastically increasing computational demands and the volumes of 
data, cluster 
management systems~\cite{borg:eurosys15,mesos:nsdi11,tupperware,yarn:socc13,
autopilot:sosp07,quincy:sosp09,apollo:osdi14,fuxi:vldb14} 
incorporate more and more applications. Applications not only include typical 
batch frameworks, but also \emph{critical} applications such as trading, fraud 
detection, health care, social-networking, and military applications. These 
applications naturally require high-availability and low performance overhead in 
deployments. For instance, a high-frequency trading application tends to be 
highly-available during its operation hours, and adding merely hundreds of \us 
to its response time means big money lost~\cite{nosql:finance}. Moreover, minor 
machine errors in a social-networking application have turned down the 
whole application site~\cite{facebook:outage}.

% This 
% paper considers an application \emph{critical} if it has high requirements on 
% availability and response time. 

% There are already enormous advances in building and applying cluster 
% management systems~\cite{borg:eurosys15,mesos:nsdi11,tupperware,yarn:socc13,
% autopilot:sosp07,quincy:sosp09,apollo:osdi14,fuxi:vldb14}. 

Unfortunately, despite much effort, existing systems are still difficult to 
meet the high requirements of critical applications because these systems do 
not provide high-availability to applications. Specifically, existing 
systems typically replicate their job controllers which accept jobs, 
allocate computational resources, and reschedule jobs on exceptional cases. 
However, the applications themselves are not replicated: if an application 
crashes or a computational resource goes down (\eg, hardware errors), these 
systems have to reschedule or recompute jobs, leaving an arbitrary 
unavailable time window for these applications.
% A possible key reason of this problem is that these systems are 
% initially designed for big-data engines which already considered fault recovery 
% (but, not availability).

% P2: SMR a promising approach, but slow.
State machine replication (SMR)~\cite{paxos} is a promising approach 
to address the availability problem of critical applications. SMR runs the 
same program on a number of replicas and uses a distributed consensus protocol 
(\eg, 
\paxos~\cite{paxos:practical,paxos,paxos:simple,paxos:complex,epaxos:sosp13}) 
to enforce the same inputs among 
replicas. An input consensus can achieve as long as a majority of replicas 
agree, thus SMR can tolerate various faults such as minor replica failures. For 
instance, two existing systems \borg~\cite{borg:eurosys15} and 
\mesos~\cite{mesos:nsdi11} use \paxos to replicate their controllers and let a 
leader controller handle all jobs. If this controller fails, \paxos elects 
another leader controller.

% Typically, \paxos assigns a replica as the leader to propose 
% consensus requests, and the other replicas agree or reject requests. 


However, \paxos's consensus is notoriously difficult to be fast. 
To agree on an input, traditional consensus protocols invoke at least one 
message round-trip between two replicas. A \v{ping} in Ethernet 
takes hundreds of \us. This is probably why existing 
systems only use \paxos to elect a leader controller, not to agree on inputs 
(\ie, job requests).

% P3: RDMA, new opportunity. Two RDMA systems.
Fortunately, Remote Direct Access Memory (RDMA) becomes commonplace in 
datacenters because of the decreasing prices of RDMA hardware. RDMA allows a 
process to directly write to the user space memory of a remote process, 
completely bypassing the remote OS kernel or CPU (the so called ``one-sided" 
operations). An evaluation~\cite{pilaf:usenix14} shows that such a write 
round-trip takes only $\sim$3 \us in the Infiniband 
architecture~\cite{infiniband}. A recent 
RDMA-enabled \paxos implementation~\cite{dare:hpdc15} reported 
$\sim$15 \us in consensus latency.

A naive approach to achieve high-availability for critical applications could 
be integrating \paxos within every application. However, this integration 
approach has two major issues. First, \paxos is notoriously difficult 
to understand~\cite{raft:usenix14}, implement~\cite{paxos:practical}, and 
verify~\cite{demeter:sosp11}.

Second, running a \paxos-enabled application in a 
cluster management system could be problematic because the system is unaware 
of replication logic. For instance, if the system happens to schedule all 
the replicas of the same job on the same machine, then a single-point failure 
of this machine will turn down all replicas. 
% Integrating \paxos to an application requires 
% strong \paxos expertise for the application developers

% As a common 
% RDMA practice, to ensure that such a write successfully resides in the 
% memory of a remote process, the local process should wait until the remote NIC 
% (network interface card) sends an ACK to the local host's NIC. 

% P4: We present a new fast, highly-available computing platform by building a 
% eco system with Mesos and RDMA SMR systems.
This paper proposes the design of \xxx,\footnote{We name our system after the 
ancient Chinese three-legged tripod, a reliable, multi-purpose container.} a 
cluster management system that automatically provides high-availability to 
general applications. \xxx maintains replicas of controllers with a new 
RDMA-enabled \paxos protocol~\cite{falcon:github}. Unlike existing systems which 
let only the leader controller schedule jobs, \xxx runs replicas of the same job 
using the replicas of controllers: it lets each controller schedule a copy of 
the same job after this job has reached consensus among controllers. 

To avoid the two aforementioned integration issues, \xxx chooses to integrate 
\paxos in the cluster system, not in applications. Doing so has two benefits. 
First, \xxx's \paxos acts as a single, general SMR service to applications. We 
can just leverage existing verification 
tools~\cite{modist:nsdi09,demeter:sosp11} to make sure that this \paxos 
protocol is robust and correct, and then we can benefit many applications. 
Second, now \xxx's own scheduler can handle the replication logic and do 
careful, replication-aware scheduling for jobs.

%  In \xxx, a 
% replicas of controllers run a critical application's same task on diverse 
% physical machines, and controllers use a fast \paxos protocol to agree on 
% computational 
% requests. 

% Benefits.
In an implementation level, \xxx integrates \mesos~\cite{mesos:nsdi11}, a widely 
used cluster management system, with a new RDMA-enabled \paxos 
protocol~\cite{falcon:github}. Compared to a prior RDMA-enabled \paxos 
protocol~\cite{dare:hpdc15}, our new protocol can support general programs 
transparently without modifications.

\xxx's \mesos-\paxos integration forms a mutual beneficial 
eco-system: by replicating jobs with \paxos, 
\xxx ensures that minor failures in a job does not affect other replicated 
jobs to compute results efficiently. By leveraging \mesos's resource allocation 
and isolation features, \xxx can run different jobs' replicas on the same 
physical machine, reducing much of the hardware cost in traditional 
\paxos approaches~\cite{crane:sosp15,rex:eurosys14}.

% P5: Applications enjoy better latency and fault-tolerance.

% P6: Benefit, now frameworks can focus on their own logic, no longer need 
% fault-toelrance module. Fault-tolerance requires expert knowledge, really 
% hard to build.

% P7: Can also address the performance problem: slow tail.
% A happy side-effect in \xxx is that it 
% addresses a pervasive latency-tail challenge via its fault-tolerance 
% architecture. 

In addition to fault-tolerance, \xxx also provides a tail-tolerance 
feature for cluster computing. The tail-of-latency distribution 
problem~\cite{tail:cacm13} (or straggler) is pervasive: because a 
job request consists of tasks running on diverse machines, the response time of 
this job is often bounded by the slowest task. Although \xxx is not the first 
work that presents a replication approach to address the straggler 
problem~\cite{dolly:nsdi13}, \xxx provides this feature in a simple and general 
way on top of its \paxos replication architecture.

% We propose a general approach with 
% \xxx's fault-tolerance architecture: \xxx collects replicas' output on a request 
% with its consensus protocol, it then replies to client programs with the 
% fastest replica's result. Therefore, applications can safely ignore this problem 
% and focus on their application logic.

% P8: in essense, a datacenter OS: handles resource allocation/isolation, 
% fault-tolerance, and performance for frameworks (applications). Use 
% replication to bstract away % physical machine details; frameworks now only 
% see computing resources, safely ignore other things such as physical 
% machines, fault-tolerance, and tails.
As part of \xxx's development, we have implemented an RDMA-enabled \paxos 
protocol~\cite{falcon:github} in Linux. This protocol intercepts a general 
program's inbound socket calls (\eg, \recv) and agrees on the received inputs 
in these calls across replicas of the same program.

To emulate a typical social-networking application, we ran this protocol with 
\redis~\cite{redis}, a popular key-value store used by Twitter. Evaluation 
showed that, compared to \redis's unreplicated execution, \xxx incurred merely a 
\tputoverhead overhead in throughput and \latencyoverhead in response time. 
This protocol was 40.1X faster than a widely used \zookeeper~\cite{zookeeper} 
SMR protocol~\cite{calvin:sigmod12} which runs on TCP/IP.
% These initial evaluation results suggest that 
% \xxx can achieve reasonable performance overhead on critical applications.

% P9: Feasibility study.

% P10: rest of the paper.
Our major conceptual contribution is \xxx, the first cluster management 
system design that provides high-availability to general applications. Our 
engineering contribution includes a fast, general RDMA-enabled \paxos protocol, 
with its source code available at~\cite{falcon:github}.

The remaining of this paper is organized as follows. \S\ref{sec:background} 
introduces background on \paxos and RDMA. \S\ref{sec:overview} gives an 
overview of our \xxx design. \S\ref{sec:tail} describes \xxx's design on 
tail-tolerance. \S\ref{sec:eval} presents evaluation results, 
\S\ref{sec:related} discusses related work, and \S\ref{sec:conclusion} 
concludes.   


