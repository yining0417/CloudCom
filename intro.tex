\section{Introduction} \label{sec:intro}

% P1: lots of applications require various big data frameworks. Trading, fraud 
% detection, aviation, medical, military. But existing systems can not meet the 
% high latency and availability requriements of these applications.
Driven the the drastically increasing computational demands and the volumns of 
data, more and more applications are being pushed to deploy in cluster 
managemement systems~\cite{mesos,borg,helix,yarn} in order to harness the 
computational resources in clusters. These applications not only include typical 
big-data frameworks (\eg,~\cite{spark}), but also critical applications such as 
trading platforms, fraud detection systems, health care systems, and military 
systems. We consider an application \emph{critical} if its high requirements on 
availability and response time. For example, a high-frequency trading platform 
tends to be highly-availabile during the stock operation hours, and adding a 
few hundred \us to the platform's response time means huge money lost.

Unfortunately, despite recent advances in building and applying cluster 
managemement systems~\cite{mesos,borg,helix,yarn}, these systems are still 
difficult to meet the high requirements of critical applications because these 
systems do not provide high-availability to the applications. Specifically, to 
make the systems themselves highly-availabile, existing systems typically 
replicate their controller component which accepts tasks, allocate 
computational resources, and (re)schedule tasks on availabile resources. 
However, the applications themselves are not replicated: if an application 
crashes or a computational resource goes down (\eg, hardware errors), these 
systems have to reschedule the tasks, leaving an arbitral unavailable time 
window for these applications.

% P2: SMR a promising approach, but slow.
State machine replication (SMR)~\cite{paxos} may be a promising conceptual idea 
to address the availability problem of critical applications.

% P3: RDMA, new opportunity. Two RDMA systems.

% P4: We present a new fast, highly-available computing platform by building a 
% eco system with Mesos and RDMA SMR systems.

% Benefits.

% P5: Applications enjoy better latency and fault-tolerance.

% P6: Benefit, now frameworks can focus on their own logic, no longer need 
% fault-toelrance module. Fault-tolerance requires expert knowledge, really 
% hard to build.

% P7: Can also address the performance problem: slow tail.

% P8: in essense, a datacenter OS: handles resource allocation/isolation, 
% fault-tolerance, and performance for frameworks (applications). Use 
% replication to bstract away % physical machine details; frameworks now only 
% see computing resources, safely ignore other things such as physical 
% machines, fault-tolerance, and tails.

% P9: Feasibility study.

% P10: rest of the paper.


