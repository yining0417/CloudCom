\section{Introduction} \label{sec:intro}



% P1: dynamic program analysis framework is good, and traditional ones are 
% synchronous.
Dynamic program analysis frameworks have become increasingly pervasive and 
critical because they enable a wide range of powerful analysis tools at 
runtime, including reliability, profliing and logging tools. 
Traditional program analysis frameworks either perform inline analysis with an 
application's actual execution (\eg, Valgrind, DynamoRio, Pin, LIFT), or 
frequently synchronize an application's execution states between the actual 
execution and the analysis (\eg, ShadowReplica), thus we call these frameworks 
\emph{synchronous} frameworks. These frameworks have embraced numerous analysis 
tools and have greatly improved software quality (\eg, detecting harmful bugs).


% P2: But performance is bad. Async approach with record-replay.
An open problem for these frameworks is performance: powerfuly analyses are 
usually heavyweight and prohibitively slow down the actual execution that 
runs synchronously. This problem has restricted most of existing frameworks as 
well as the analysis tools to deploy only in testing labs without being able 
to exercise real-world workloads. In fact, many analyses can be done 
asynchronously (\eg, data race detection and profling tools), so recently 
researchers have proposed asynchronous frameworks (\eg, Aftersight, Spec, 
ASPLOS13) via an advanced technique called ``record-replay": the record 
phase runs the actual execution and records nondeterministic sources such as 
inputs, while the replay phase runs (typically offline) the actual execution 
with an analysis tool on the recorded inputs. This asynchronous approach 
decouples the analysis from the execution and lets the actual execution run 
efficiently.


% P3: Multithreading makes record-replay bad.
Unfortunately, the emerging multithreading trend, which is driven by the rise 
of the multi-core hardware, poses a significant challenge on existing 
asynchronous frameworks. For instance, today's popular server applications 
(\eg, \apache) use multiple threads to server requests parallelly. These 
parallel and continuously running applications require record-replay to run 
online (cite Respec) which frequently synchronize thread interleavings (or 
\emph{schedules}), otherwise the record and replay may run into different 
schedules and easily leading to divergent execution states. Despite a variety 
of notable approaches are proposed (cite Respec, DoublePlay, and ASPLOS 2013), 
efficient and scalable online record-replay is still an open problem. For 
instance, although a recent work shows that they can parallize data race 
detection by 4X after applying some smart optimizations, they admits that the 
detection has a total 20X to 60X slowdown with their benchmarks).


Worse, existing optimizations significantly trade off transparency between the 
analyses and the frameworks: for better performance, they significantly orchest 
the frameworks with specific type of analysis, which requires reconstruct from 
the analysis (two race detection algorithms) into three phases in their 
framworks, although the analysis algorithms themselves are already notouriously 
complicated and subtle. Existing synchronous frameworks' performance also 
suffer from multithreading (cite ShadowReplica).


% P4: Our approach. Replication. DMT. And why we are better than existing 
% replication systems: avoid shipping schedules, avoid annotations.
This paper presents \xxx, an efficient, transparent program 
analysis framework for asynchronous analyses. The key technique we leverage is 
state machine replication (or \smr). As a typical \smr setting to keep the 
replicas processing the same sequence of input requests, \xxx incorporates a 
distributed consensus protocol \paxos~\cite{paxos} so that as long as a 
majority of nodes aggree on a new request, all replicas comply with this 
agrreement. \xxx runs replicas of an application on a number of 2\v{f}+1 
machines, at least \v{f}+1 machines (majority) run the actual executions, agree 
on the input requests, and respond to client programs fast; a number of 1 to 
\v{f} replicas can run one type of analysis on each. To keep 
schedules across replicas in sync, \repbox leverages an advanced techniqued 
called Deterministic Multithreading (or \dmt). Recent 
study~\cite{parrot:sosp13} has shown that \dmt runs fast on a wide range of 108 
popular multithreaded programs on 24-core machines. In sum, the \smr 
replication enables the actual executions proceed efficiently, and \dmt 
eliminates the burden of synchronizing schedules across replicas as in 
record-replay.


% P5: Benefits.
\xxx has three major benefits. First, \xxx's replication architecture provides 
an efficient framework for many asynchronous analysis tools, ranging from 
reliability tools, to profiling tools, and to logging tools. Second, the 
flexibility on the number of replicas enable multiple analyses at the same 
time. No prior evaluation shows that it can support multiple analyses 
simultaneously. Third, unlike many previous work which tightly orchest the 
framework with the analysis for better performance (\eg, ShadowReplica and 
ASPLOS13), our framework is independent to analysis logic, so it is 
complementary to existing analyses as well as frameworks, thus \xxx can easily 
deploy existing frameworks in our replicas (\eg, Valgrind).


% P6: A practical challenge. Transparent checkpoint.
However, to make \xxx's asynchronous analysis framework practical, two 
challenges should be addressed. First, depending on machine restarts and 
network partitions, a node's role may change, not only between a \paxos primary 
and backup, but also between normal nodes and analysis nodes, ande even between 
different analysis nodes. We design a checkpoint mechanism that matches up 
checkpoints from different role machines with input total order ID to address 
this challenge. Second, we need a mechanism to roll back all replicas when a 
bad event (\eg, deferencing a NULL pointer) has occured and detected by the 
analysis tool. We design a a simple API for analysis tool to invoke so that the 
replicas can reach consensus on this roll back and act consistently.




% P2: Existing analysis frameworks have two problems because they are 
% synchronous. Reliability, Securityk, and Profling tools.
% First, slow. Second, can support only one analysis at the same time. Three, 
% % need to orchest the analysis algorithm as well as the framework to adapt to 
% the framework.
% 1: slow.

% 2: can not support mulitple analysis at the same time.
%   Can valgrind do multiple analysis at the same time?
%   Can Speck do multiple analysis at the same time?

% 3: need to orchest the analysis significantly, or even sacrifice guarantees.
%   Peter Chen's: need to separate anaysis to multiple stages.
%   TaintDroid: sacrifice control flow taint.  
% Unfortunately, despite decades of effort, existing program analysis 
% frameworks 
% % are still extremely difficult to deploy in production runs, mainly due to 
% two 
% % % problems. First, these frameworks prohibitively slow down the actual 
% % executions % (\eg, up to 40X in Valgrind and 12X in ShadowReplica), because 
% the % synchronous % approach in these frameworks need to frequently exchange 
% program % states between % the executions and the analysis tools. Whenever a 
% piece of % analysis work is % slow, the actual execution has to wait for the 
% analysis to be % done.
% 
% Second, the synchronous approach causes existing frameworks to significantly 
% involve with the program state exchange with the analysis tool, causing these 
% frameworks to support only one analysis tool at the same time. To the best of 
% our knowledge, no evaluation in existing frameworks have shown to be able to 
% support multiple analysis tools. This problem leads to a paradox: if an 
% application wants the benefit of one analysis tool, it has to exclude the 
% other % tools.

% P3: Why existing sync approach must fail.
% We argue that the synchronous approach in existing frameworks is not 
% fundamental, and many anlysis tools (\eg, data race detectors and profiling 
% tools) can be ran asynchronously, if there is we could meet these 
% requirements: (1) there is a transparent replication framework that can 
% replicate inputs and nondeterministic events such as threads aquiring locks; 
% and (2) there is a mechanism for a analysis tool to notify the framework if 
% bad % events such as deferencing a null pointer occurs or the tool detects 
% that % a % stack overflow occurs; (3) the framework has a transparent 
% application % checkpoint and recovery feature if bad events occurs. Even for 
% a % security % analysis tool, if the security threat does not involve leaking 
% information to % the outside world (\eg, using deallocated memory or 
% uninitialized memory), % which application checkpoint can not revert, this 
% asynchronous approach could % be suitable for this tool.


% P4: Our key insight is that many analysis does not have to be synchronous, 
% and % they can be implemented asynchrously. This paper presents asynchronous 
% analysis % fvalgrind --tool=memcheck ls -lramework. Key weapon, transparent
% state machine replication.

%% Existing types of analysis frameworks:
% 1: shadow memory. Valgrind. TaintDroid. YY Zhou's Lift. Pin. DynamoRio.
% 2: record replay. Peter Chen's Aftersight.
% 3: decoupling execution and analysis. ShadowReplica. Speck (also have record 
% replay).

% P5: Application scope. Not for leaking analysis.
% To meet these requiremnts, this paper presents \xxx, a program analysis 
% framework that supports asynchronous and transparent program anlysis tools. 
% The % core of \xxx is \repbox, a state machine replication (or \smr) system 
% that % can % transparently replicate today's general multithreaded programs 
% for % high % availability. \smr runs replicas of an application on multiple 
% machines % (nodes), tolerating many possible node and network failures.  To 
% keep % the % replicas consistent, it invokes a distributed consensus protocol 
% \paxos~\cite{paxos}) to ensure that a majority of the replicas agree on the 
% input request sequence. \repbox incorporates an efficient Deterministic 
% Multithreading (or \dmt) engine, which practically enforces the same thread 
% interleavings for the application running across replicas.
% 
% \xxx addresses both the two problems in today's analysis frameworks. A 
% typical % setting of a \xxx deployment contains 2\v{f}+1 nodes. A node acts 
% as % the % primary node, which accepts client requests, invokes \paxos 
% consensus on % the % order of these requests, and sends back responses to 
% clients. The other % nodes % act as backup nodes, which reach consensus on 
% requests and process % requests. % One or multiple analysis tools run on at 
% most % \v{f} nodes, while the % other % nodes run the application without 
% analysis and % process requests fast.
% 
% However, to make \xxx's asynchronous analysis framework practical, two 
% practical challenges must be addressed. First, depending on machine restarts 
% and network partitions, a node's role may change, not only between primary 
% and % backup, but also between normal nodes and analysis nodes, ande even 
% between % different analysis nodes. We design a new checkpoint mechanism that 
% matches up % checkpoints from different role machines with input total order 
% ID % to address % this problem. Second, we need a mechanism to roll back all 
% replicas % when a bad % event (\eg, deferencing a NULL pointer) has occured 
% and % detected by % the % analysis tool. We design a a simple API for 
% analysis % tool % to invoke so % that the % replicas can reach consensus on 
% this % roll back % and % act consistently.
% 
% To verify that \xxx's asynchronous approach is feasible, we have evaluated 
% \xxx with four diverse types of widely uses server programs. We measured the 
% checkpint time cost, run multiple valgrind analysis on different nodes, and 
% collected performance results. These results also show that \xxx is 
% complementary to existing analysis frameworks.

% Problem: for some security attack such as stack overflow, the attacker 
% may totally replace the LD_PRELOAd library so that our interception layer
% can never go back, and we can never roll back.

% P6: Key technical challenges. Transparent and timely (to respond to attacks) 
% checkpoint. Migration of analysis from one machine to another.
% When bad events are detected, how does the analysis involve with the 
% framework?
% Notify the framework, involve sync point. Roll back the replicas.

% P7: Our initial results. Checkpoint time. Run with valgrind one analysis. Or 
% two valgrind analysis, one at each replica.
