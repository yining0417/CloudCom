\section{Introduction} \label{sec:intro}

% P1: virtualization is good
By simplifying provisioning and allowing multiple servers to be consolidated on a 
small number of physical hosts, virtualization has made low- and middle-end systems 
popular than ever. Nevertheless, the benefits of consolidation come with a hidden cost 
in the form of increased exposure to hardware failure. 

% P2: However, previous approaches still fail to efficiently provide fault-tolerance in virtualized environment
Although recent advances in accommodating virtual machines with high availability are encouraging, 
existing works are still unable to ensure fault-tolerance in virtualized environment at low overhead. 
Currently there are two major techniques to provide fault tolerance.

% P3: log-replay is highly architecture-specific and slow for multi-processor
Log-replay records input and non-deterministic events of the primary machine 
and have them deterministically replayed on the backup machine to replicate the primary's state 
in case the primary node fails. However, deterministic replay is highly architecture-specific and 
introduces unacceptable performance degradation when applied in multi-core systems.

% requiring that the system have a comprehensive understanding of the instruction set being executed and the sources of external events
% reproducing the exact order in which CPU cores access the shared memory

% P4: Checkpoint-recovery based replication of virtual machines, it suffers from two problems
% 1. significant VM downtimes
% 2. network delay. unacceptable for online services where clients expect short response times from the server machines
Checkpoint-recovery based replication of virtual machines is the other commonly used solution. 
It captures the entire execution state of the running VM at relatively high frequency so that changes 
can be reflected to the backup machine nearly instantly. However, it suffers from two problems.

% P4: Our key insight is that instead of capturing snapshots of the running VM and propagating changes frequently, 
% enforcing the same ordering over network interaction from outside the VM at all nodes is sufficient to ensure fault tolerance
% Paxos, a consensus protocol, can help us achieve this

% Note: limitation: cannot handle non-deterministic events.


% P6: However, traditional Paxos is slow
% Fortunately, RDMA opens new opportunity for mitigating consensus latency
However, Paxos consensus is notoriously difficult to be fast. 
To agree on an input, traditional consensus protocols invoke at least one 
message round-trip between two replicas. Fortunately, Remote Direct Access Memory (RDMA) opens new opportunity.

% P7: Besides, integrate Paxos within each application is notoriously difficult because it is difficult to understand and implement.
% We choose to intercept the network syscalls within the hypervisor.

% P8: We present a fast, fault-tolerant system.
% We intercept at the networking level and invoke RDMA-based Paxos protocol.

% P9: benefits: fast (short response time), service is always available (VM does not need to be suspended)

% P10: rest of the paper
