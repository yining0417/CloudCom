\section{Introduction} \label{sec:intro}

% implement application fault tolerance at the virtual machine level

% P1: virtualization is good
Server virtualization has emerged as a powerful technique for consolidating servers in 
data centers. The reduction of the number of physical hosts also contributes to cutting 
back the power consumptions in the data centers.

% Symantic proposes a solution, based on VMware products, where the applications are monitored,
% however this commercial solution applies for a specific list of supported applications, 
% and does not support HA in a generic way

At the same time, a failure of a hosting server becomes a serious problem in consolidated 
server systems using virtualization. There has been a tremendous progress in realizing fault 
tolerance in virtualized environments, but existing approaches still fail to meet the high 
availability requirements of applications running inside the VM because they are oriented 
toward protecting the VM. The VM alone does not guarantee the availability at the application 
level. Detecting and remediating VM failure falls short of what is truly vital, detecting and 
remediating application and service failures.

State Machine Replication (SMR) is an attractive approach to achieving application fault tolerance. 
SMR runs replicas of the program and invokes a distributed consensus protocol 
(typically \paxos) to ensure the same sequence of input requests for replicas, as long as a 
quorum (typically a majority) of the replicas agrees on the input request sequence.

However, \paxos is notoriously slow because each decision takes at least 
three message delays between when a replica proposes a command and when some replica learns which 
command has been chosen.

Fortunately, Remote Direct Memory Access (RDMA)-capable networks have dropped in price and made 
substantial inroads into datacenters. RDMA operations allow a machine to read (or write) from a 
pre-registered memory region of another machine without involving the CPU on the remote side. 
Compared to traditional message passing, RDMA achieves the smallest round-trip latency 
(~$\sim$3 \us), highest throughput, and lowest (zero) CPU overhead~\cite{pilaf:usenix14}.

% This paper presents \xxx, a software system that provides high availability

The remainder of the paper is organized as follows.
