\section{Checking Network Outputs} \label{sec:output}

This section presents \xxx's network outputs checking mechanism for 
detecting and recovering from replicas' divergence of execution states.

To capture network outputs of the server programs running in the virtual machines, we 
modified the \taprecv function in QEMU to maintain a packet queue for each application 
that captures the corresponding outgoing packets. The network outputs are pushed into the 
queue and whenever the queue is full, a new hash value is calculated by 
$h_i=H(h_{i-1}\|H(queue))$ where $H()$ is a hash function and $||$ stands for concatenation. 
Such a computation links the hash value to all the previous network outputs. Then, after 
every \thashcomp hash values are generated, the latest one is passed to \smrsystem and 
the output checking protocol is invoked. The index of this hash value in the hash chain is 
consistent across replicas because each replica implements the same mechanism. 

% To make KVM_GET_DIRTY_LOG operation consistent between replicas
%  ------------------------   ------------
% | KVM_GET_DIRTY_LOG entry| | hash entry |
%  ------------------------   ------------

Once divergence of execution states is detected, we do a light-weight state transfer to 
handle it. For memory states, we make use of the \emph{dirty page log} facility provided 
by \kvm. The \kvm kernel subsystem interfaces \emph{KVM\_SET\_USER\_MEMORY\_REGION} and 
\emph{KVM\_GET\_DIRTY\_LOG} allow the caller to keep track of dirty page state changes 
for the given virtual machine. KVM\_SET\_USER\_MEMORY\_REGION allows the user to create or 
modify a guest physical memory slot, and KVM\_GET\_DIRTY\_LOG returns a bitmap with all 
dirty pages since the last call. With these APIs, we can compare dirty memory pages and 
sync the different ones only.