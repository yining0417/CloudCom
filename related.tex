\section{Related Work} \label{sec:related}

\para{Cluster Management Systems.} TBD.  

\para{State machine replication (SMR).}  SMR is a powerful, but 
complex fault-tolerance technique. The literature has developeed a rich set of
\paxos 
algorithms~\cite{paxos:practical,paxos,paxos:simple,paxos:complex,epaxos:sosp13}
and implementations~\cite{paxos:live,paxos:practical,chubby:osdi}. \paxos is 
notoriously difficult to be fast and scalable~\cite{ellis:thesis}. To improve 
speed and scalability, various advanced replication models have been 
developed~\cite{epaxos:sosp13,mencius:osdi08,scatter:sosp11,manos:hotdep10}. 
Since consensus protocols play a core role in 
datacenters~\cite{matei:hotcloud11, mesos:nsdi11, datacenter:os} and 
distributed 
systems~\cite{spanner:osdi12,mencius:osdi08}, a variety of study have been 
conducted to improve different aspects of consensus protocols, including 
performance~\cite{epaxos:sosp13,paxos:fast,dare:hpdc15}, 
understandability~\cite{raft:usenix14,paxos}, and verifiable reliability 
rules~\cite{modist:nsdi09,demeter:sosp11}. Although \xxx tightly integrates 
RDMA features in \paxos, its implementation mostly complies with a popular, 
practical approach~\cite{paxos:practical} for reliability. Other \paxos 
approaches can also be leveraged in \xxx.

\para{RDMA techniques.} RDMA techniques have been implemented in various 
architectures, including Infiniband~\cite{infiniband}, RoCE~\cite{roce}, and 
iWRAP~\cite{iwrap}. RDMA have been leveraged in many systems to improve 
application-specific latency and throughput, including high performance 
computing~\cite{openmpi}, key-value 
stores~\cite{pilaf:usenix14,herd:sigcomm14,farm:nsdi14,memcached:rdma}, 
transactional processing systems~\cite{drtm:sosp15,farm:sosp15}, and file 
systems~\cite{gibson:nfs}. These systems are largely complementary to \xxx.
% It 
% will be interesting to investigate whether \xxx can improve the availability 
% for 
% both the client and server for some of these advanced systems within a 
% datacenter, and we leave it for future work.